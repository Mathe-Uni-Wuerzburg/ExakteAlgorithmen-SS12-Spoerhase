\chapter{Measure \& Conquer}

\section{Maximal Independent Set}
  Dabei handelt es sich um eine Methode, um die Laufzeit von Branching-Algorithmen zu analysieren. Bisher war das Maß für den Fortschritt die Eingabegröße (Anzahl der Knoten, Anzahl der Elemente, etc.). Jetzt erhalten wir durch ``beliebige'' Maße mehr Wahlfreiheit und erreichen so bessere Schranken für die Laufzeit. Viele der heute schnellsten Algorithmen sind Branching-Algorithmen, die mit Measure \& Conquer analysiert wurden. 
  
  Die Anforderungen an das Maß lauten wie folgt:
  \begin{itemize}
   \item Das Maß ist nach Problemgrößenreduktion durch das Branching kleiner als vorher.
   \item Das Maß ist nicht negativ.
   \item Das Maß ist von oben durch eine Funktion der Problemgröße beschränkt. (Rückführung in Laufzeitschranke über $n$.)
  \end{itemize}

  

  \begin{algorithm}[H]
    \caption{Algorithmus zur Berechnung der größten Anzahl unabhängiger Mengen}
  
    \KwName{MIS} \\
    \KwData{Graph \(G\)}
    \KwResult{Größte Anzahl von unabhängigen Mengen}

    \uIf{ex. \(v\) mit \(d(v) =0\)}{
      \Return 1 + MIS(\(G-v\)) (1)
    }\uElseIf{ex. \(v\) mit \(d(v) =1\)}{
      \Return 1 + MIS(\(G-N[v]\)) (2)
    }\uElseIf{ \(\Delta(G) \geq 3\) } {
      wähle \(v\) mit \(d(v) \geq 3\) \\
      \Return \(\max \{ 1 + \text{MIS}(G - N[v]), \text{MIS}(G-v) \}\) (3)
    }\ElseIf{ \(\Delta(G) \leq 2\) } {
      löse diesen Spezialfall effizient (4)
    }
  \end{algorithm}

  Nur in Fall (3) handelt es sich um Branching mit Branching-Vektor $(1,d(v) +1) \vartriangleright (1,4)$. Die Laufzeit ist hier also $O^*(\tau(1,4)^n) = O^*(1{,}3803^n)$. Diese Analyse kann durch die Methode Measure \& Conquer verbessert werden: Nur Knoten mit $d(v) \geq 3$ führen zum Branching. Der Ansatz für unser Maß lautet also, nur die Konten $k_1(G) = n_{\geq 3}$ -- also nur die Knoten mit Grad größer-gleich 3 -- zu zählen. Der Branching-Vektor ist also $(1,1)$, daraus ergibt sich im Worst-Case eine Laufzeit von $O^*(2^n)$. 
  
  AKTUELL BIS HIER!
  

\section{Dominating Set}
  Wir betrachten nun einen Algorithmus für Dominating Set, der bei konventioneller Analyse Laufzeit \(O(1{,}9052^n)\), bei Analyse mit Measure \& Conquer jedoch \(O(1{,}5259^n)\) ergibt. Dazu modellieren wir Dominating Set als Set Cover-Instanz, wobei \(U = V\) und \(S = \{ N[v] : v \in V \}\) ist. Die Größe dieser Set Cover-Instanz ist dann \(|U| + |S| = n + n = 2n\). Wir entwickeln daher einen Algorithmus für Set Cover mit \(O^*(c^{|U|+|S|})\) und \(c \ll 2\).

  Wir nehmen ohne Einschränkung an, dass \(U = \bigcup_{S_i \in S} S_i\). Damit ist eine Set Cover-Instanz durch \(S\) spezifiziert. Die \defNotion{Häufigkeit} eines Elements \(u \in U\) sei definiert als die Anzahl der Mengen aus \(S\), die \(u\) enthalten.

  \begin{algorithm}[H]
    \caption{Algorithmus für Set Cover}

    \KwName{SC} \\
    \KwData{Set Cover-Instanz \(S\)}
    \KwResult{}

    \uIf{(1) \(|S| = 0\)}{
      \Return 0
    }\uElseIf{(2) es existieren \(S_1, S_2 \in S\) mit \(S_1 \subseteq S_2\)}{
      \Return SC(\(S - S_1\))
    }\ElseIf{(3) es existiert \(u \in U(S)\) mit Häufigkeit 1, d.h. liegt in genau einem \(S^* \in S\) } {
      \Return 1 + SC(del(\(S^*, S\)))
    }
    wähle Menge \(S' \in S\) mit maximaler Kardinalität \;
    \uIf{(4) \(|S'| = 2\) }{
      löse in Polynomialzeit (vgl. Übungsaufgabe) \;
    }\ElseIf{ \(|S| \geq 3 \) } {
      \(S_{in} = S - S^*\)\;
      \(S_{out} = \text{del}(S, S^*)\)\;
      \Return \(\min\{ \text{SC}(S_{in}), 1 + \text{SC}(S_{out}) \}\)
    }
  \end{algorithm}

  Dabei ist \(\text{del}(A,S) = \{ T : T = R \setminus A \neq \emptyset, R \in S \}\).

  Für die konventionelle Analyse sei \(k'(S) = |S| + |U(S)|\). Wir können aufgrund des Algorithmus die Rekursionsungleichung
  \[ T(k') \leq T(k' - 1) + T(k' - 4) \]
  herleiten. Die übliche Lösungsmethode ergibt \(T(k') = O(1{,}3803^{k'})\). Damit ergibt sich als Laufzeit für Dominating Set \(O(1{,}3803^{2n}) = O(1{,}9052^n)\).

  Für die Analyse mittels Measure \& Konquer definiert man \(n_i\) als Anzahl der Mengen mit Kardinalität \(i\) und \(m_j\) als Anzahl der Elemente mit Häufigkeit \(j\). Das Maß sei
  \[ k(S_i) = \underbrace{ \sum_{i \geq 1} w_in_i }_{k_w(S_i)} + \sum_{j \geq 1} v_jm_j \]
  mit Gewichten \(w_i, v_j \in [0,1]\). Damit ist \(k(S_i) \leq |U| + |S|\). Vereinfachend nehmen wir für die Gewichte an, dass
  \begin{enumerate}[(a)]
   \item \(w_i \leq w_{i+1}, v_i \leq v_{i+1}\),
   \item \(w_1 = v_1 = 0\),
   \item \(w_1 = v_1 = 1\) für \(i \geq 6\),
   \item \(\Delta w_i \geq \Delta w_{i+1}\) mit \(\Delta w_i := w_i - w_{i-1}\) und \(\Delta v_i\) analog.
  \end{enumerate}
  Wir bestimmen den Branching-Vektor \( (\Delta k_{out}, \Delta k_{in}) \) in Abhängigkeit der \(w_i\) und \(v_i\).
  \begin{enumerate}[(1)]
   \item Im Fall, dass \(S_i\) nicht gewählt wird, verändert sich \(\Delta k_{out}\) wie folgt.
          \begin{enumerate}[(i)]
            \item Durch Wegfallen von \(S_i\) wird \(k_w(S)\) um \(w_{|S_i|}\) reduziert.
            \item Sei \(r_j\) definiert als Anzahl der Elemente von \(S_i\) mit Häufigkeit \(j\). Die Reduktion von \(k_v(S)\) ist dann kleiner oder gleich \(\sum_{i=2}^6 r_i \Delta v_i\).
            \item Angenommen \(r_2 > 0\). Seien \(R_1, ..., R_h\), \(h \leq r_2\) die Mengen ungleich \(S_i\), die mindestens ein Element der Häufigkeit 2 mit \(S_i\) teilen. Die Mengen \(R_i\) werden nachfolgend durch Regel (3) reduziert. Sei dazu \(v_{2,i}\) die Anzahl solcher Elemente in \(R_i\). Dann ist \(|R_i| \geq v_{2,i}+1\) und \(|S_i| \geq v_{2,i} + 1\). Die Reduktion von \(k_w(S)\) durch Wählen von \(R_i\) ist dann mindestens \(W_{v_{2,i}+1}\). Damit existiert mindestens ein Element, das nicht in \(S_i\) liegt und aus der Instanz entfernt wird. Also wird \(k_v(S)\) mindestens um \(v_2\) reduziert.
          \end{enumerate}
          Wir erhalten eine Reduktions von mindestens
          \[
           \Delta k' =
           \begin{cases}
             0 & (r_2 = 0), \\
             v_2 + w_2 & (r_2 = 1), \\
             v_2 + w_2 + w_3 & (r_2 = 2), \\
             v_2 + w_4 & (r_2 \geq 3, |S_i| = 3),
           \end{cases}
          \]
          Demnach ist in diesem Fall \(\Delta k_{out} = w_{|S_i|} + \sum_{i=2}^6 r_i \Delta v_i + \Delta k'\).
   \item Im Fall, dass \(S_i\) gewählt wird,
          \begin{enumerate}[(i)]
           \item wird \(k_w(S)\) durch Wegfall von \(S_i\) um \(w_{|S_i|}\) reduziert,
           \item wird \(k_v(S)\) durch Wegfall von \(S_i\) um \(\sum_{i=2}^6 r_iv_i + r_{\geq 7}\) reduziert,
           \item wird \(k_w(S)\) durch die Verkleinerung der Kardinalität derjenigen Mengen, die \(S_i\) schneiden, reduziert. Sei \(R\) eine solche Menge, d.h. \(R \cap S \neq \emptyset\). Sei \(u\) also ein Element aus \(R \cap S\), dann ist der Beitrag von \(u\) zur Verkleinerung der Kardinalität größer oder gleich \(\Delta w_{|R|} \geq \Delta w_{|S_i|}\). Damit beträgt die gesamte Reduktion für diesen Beitrag mindestens \[ \Delta w_{|S_i|} \left( \sum_{i=2}^6 (i-1)r_i + 6 \cdot r_{\geq 7} \right). \]
          \end{enumerate}
         Für den Fall, dass \(S_i\) gewählt wird beträgt die Gesamtreduktion also mindestens
         \[
           \Delta k_{in} = w_{|S_i|} + \sum_{i=2}^6 r_iv_i + r_{\geq 7} + \Delta w_{|S_i|} \left( \sum_{i=2}^6 (i-1)r_i + 6 \cdot r_{\geq 7} \right).
         \]
  \end{enumerate}

  Sei nun für feste \( (w,v) := (w_1,...,w_6,v_1,...,v_6)\) und \(|S| \geq 3\) und \((r_i)_i\) mit \(\sum_{i=2}^6 r_i + r_{\geq 7} = |S|\) die Rekurenz \(T(k) \leq T(k-\Delta k_{out}) + T(k - \Delta k_{in})\) definiert.

  Wir beobachten, dass wir für jeden Branchingvektor mit \(|S_i| \geq 7\) einen Branchingvektor mit \(|S_i| = 7\) finden, der diesen dominiert. Dies folgt aus den Formeln für \(\Delta k_{in}\) und \(\Delta k_{out}\) zusammen mit der Tatsache, dass \(\Delta w_{|S_i|} = 0\) für \(|S_i| \geq 7\). Wir nehmen daher an, dass \(3 \leq |S_i| \leq 7\) und damit ist die Anzahl der gültigen Belegungen für \(|S_i|\) und \((r_i)_i\) endlich. Also ist für jedes feste \((w,v)\) die Laufzeit beschränkt durch \(\alpha(w,v)^k\), wobei \(\alpha(w,v)\) die größte der Nullstellen der Gleichungen \(x^t - x^{t-\Delta k_{out}} - x^{t - \Delta k_{in}} = 0\) für \(t := \max \{ \Delta k_{in}, \Delta k_{out} \}\) über alle Belegungen für \(|S_i|,(r_i)_i\). Für jedes \((w,v)\) erhalten wir so eine Laufzeitschranke \(\alpha(w,v)^k\). Wir wollen nun ein \((w,v)\) finden, sodass \(\alpha(w,v)\) möglichst klein ist. Mit Methoden der quasi-konvexen Optimierung lässt sich eine Näherunglösung finden, diese liefert \(\alpha(w_0,v_0) < 1,2353\).

  \begin{theorem}
    Set-Cover lässt sich in \(O(1,2353^{|U|+|S|})\) lösen.
  \end{theorem}

  \begin{corollary}
    Dominating-Set lässt sich in \(O(1,5259^n)\) lösen.
  \end{corollary}




